\documentclass[arhiv]{../izpit}
\usepackage{amssymb}
\usepackage{fouriernc}

\begin{document}

\newcommand{\bnfis}{\mathrel{{:}{:}{=}}}
\newcommand{\bnfor}{\;\mid\;}
\newcommand{\fun}[2]{\lambda #1. #2}
\newcommand{\conditional}[3]{\mathtt{if}\;#1\;\mathtt{then}\;#2\;\mathtt{else}\;#3}
\newcommand{\whileloop}[2]{\mathtt{while}\;#1\;\mathtt{do}\;#2}
\newcommand{\recfun}[3]{\mathtt{rec}\;#1\;#2. #3}
\newcommand{\boolty}{\mathtt{bool}}
\newcommand{\intty}{\mathtt{int}}
\newcommand{\funty}[2]{#1 \to #2}
\newcommand{\tru}{\mathtt{true}}
\newcommand{\fls}{\mathtt{false}}
\newcommand{\tbool}{\mathtt{bool}}
\newcommand{\tand}{\mathbin{\mathtt{and}}}
\newcommand{\tandalso}{\mathbin{\mathtt{andalso}}}
\newcommand{\imp}{\textsc{imp}}
\newcommand{\skp}{\mathtt{skip}}
\makeatletter
\newcommand{\nadaljevanje}{\dodatek{\newpage\noindent\emph{(\@sloeng{nadaljevanje rešitve \arabic{naloga}. naloge}{continuation of the answer to question \arabic{naloga}})}}}
\makeatother
\izpit
  [ucilnica=203,naloge=-1]{Teorija programskih jezikov: 1. izpit}{14.\ junij 2022}{
}
\dodatek{
  \vspace{\stretch{1}}
  \begin{itemize}
    \item \textbf{Ne odpirajte} te pole, dokler ne dobite dovoljenja.
    \item Zgoraj \textbf{vpišite svoje podatke} in označite \textbf{sedež}.
    \item Na vidno mesto položite \textbf{dokument s sliko}.
    \item Preverite, da imate \textbf{telefon izklopljen} in spravljen.
    \item Čas pisanja je \textbf{180 minut}.
    \item Doseči je možno \textbf{70 točk}.
    \item Veliko uspeha!
  \end{itemize}
  \vspace{\stretch{3}}
  \newpage
}

%%%%%%%%%%%%%%%%%%%%%%%%%%%%%%%%%%%%%%%%%%%%%%%%%%%%%%%%%%%%%%%%%%%%%%%

\naloga[\tocke{15}]
\newcommand{\ifthen}[3]{\mathbf{if} \, #1 \, \mathbf{then} \, #2 \, \mathbf{else} \, #3}

V $\lambda$-računu z neučakano operacijsko semantiko definirajmo:
\[
  \mathit{switch} = \fun{p} \fun{f} \fun{g} \fun{x} \ifthen{p \, x}{f \, x}{g \, x}
\]

\podnaloga Zapišite vse korake v evalvaciji izraza $\mathit{switch} \, (\fun{x} x < 1337) \, (\fun{x} x + 4) \, (\fun{x} x * 11) \, 38$ v semantiki malih korakov. Izpeljav korakov ni treba pisati.
\podnaloga Izračunajte najbolj splošen tip izraza $\mathit{switch}$.
\nadaljevanje

%%%%%%%%%%%%%%%%%%%%%%%%%%%%%%%%%%%%%%%%%%%%%%%%%%%%%%%%%%%%%%%%%%%%%%%

\naloga[\tocke{20}]
\newcommand{\natty}{\mathbf{nat}}
\newcommand{\zro}{\mathbf{O}}
\newcommand{\suc}[1]{\mathbf{S} \, #1}
\newcommand{\fold}[3]{\mathbf{fold} \, #1 \, #2 \, #3}
\newcommand{\intsym}[1]{\langle #1 \rangle}
V $\lambda$-račun dodamo naravna števila:
\begin{align*}
  \text{tip } A &\bnfis \cdots \mid
    \natty \\
  \text{izraz } M, N &\bnfis \cdots \mid
    \zro \mid
    \suc{N} \mid
    \fold{M_o}{M_s}{N}
\end{align*}
kjer zadnji izraz vrne enako vrednost kot
\[
  f \, \intsym{n - 1} \, \bigg(f \, \intsym{n - 2} \, \big(\cdots (f \, \intsym{0} \, v) \cdots\big)\bigg),
\]
če izraz $M_o$ predstavlja vrednost $v$, izraz $M_s$ funkcijo dveh argumentov $f$, izraz $N$ pa število $\intsym{n} := \underbrace{\mathbf{S} \cdots \mathbf{S}}_n \mathbf{O}$.

\podnaloga Zapišite dodatna pravila za določanje tipov in operacijsko semantiko malih korakov.
\podnaloga V razširjenem jeziku definirajte funkcijo, ki sešteje dve naravni števili.
\podnaloga Dokažite, da za razširjeni jezik še vedno velja izrek o varnosti.

\nadaljevanje

%%%%%%%%%%%%%%%%%%%%%%%%%%%%%%%%%%%%%%%%%%%%%%%%%%%%%%%%%%%%%%%%%%%%%%%

\naloga[\tocke{15}]

Naj bodo $(D_1, \leq_1)$, $(D_2, \leq_2)$ in $(D, \leq)$ domene. Dokažite, da je preslikava
\[
  \mathit{curry} : [D_1 \times D_2 \to D] \to [D_1 \to [D_2 \to D]],
\]
podana s predpisom
\[
  \mathit{curry}(f) = x \mapsto (y \mapsto f(x, y)),
\]
dobro definirana in zvezna.

\nadaljevanje

%%%%%%%%%%%%%%%%%%%%%%%%%%%%%%%%%%%%%%%%%%%%%%%%%%%%%%%%%%%%%%%%%%%%%%%

\naloga[\tocke{20}]
\newcommand{\orop}{\mathrm{or}}
\newcommand{\bind}{\mathop{>\!\!\!>\!\!\!=}}

Nedeterministično izbiro opišemo z dvojiško operacijo $\orop : 2$, ki zadošča enačbam:
\[
  \orop(x, x) = x \qquad \orop(x, y) = \orop(y, x) \qquad \orop(x, \orop(y, z)) = \orop(\orop(x, y), z)
\]
Naj bo $T$ monada, porojena s prostim modelom za opisano algebrajsko teorijo. Dokažite, da velja izomorfizem množic
\[
  T X \cong \mathcal{P}_{\text{fin}} X = \{ A \mid A \subseteq^{\text{končna}} X \}.
\]

\nadaljevanje

\end{document}